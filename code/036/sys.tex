\documentclass[10pt,a4paper]{article}
\usepackage[utf8]{inputenc}
\usepackage[T1]{fontenc}
\usepackage{graphicx}
\author{Señales y Sistemas - 2023}
\title{Trabajo Final}
\date{Julio 20, 2023}
\begin{document}
	\maketitle
	\section{Definiciones}
	Repasar las siguientes definiciones
	\begin{enumerate}
		\item Señal
		\item Sistema
		\item Tiempo continuo y tiempo discreto
		\item Impulso unitario
		\item Escalón unitario
		\item Señales períodicas
		\item Sistemas con memoria y sin memoria
		\item Sistemas lineales
		\item Convolución (fórmulas en tiempo continuo y tiempo discreto), respuesta al impulso.
		\item Series de Fourier (fórmulas en tiempo continuo y tiempo discreto), propiedades.
		\item Transformada de Fourier en tiempo continuo (fórmulas de análisis y síntesis).
		\item Transformada de Laplace (fórmulas de análisis y síntesis).
		\item Explicar con ejemplos como se realiza la descomposición en fracciones parciales.
	\end{enumerate}

    \section{Ejercicios}
    \begin{enumerate}
    	\item Traer en papel (manuscrito o impreso) el laboratorio realizado.
    	\item Traer en papel (manuscrito o impreso) el cálculo de los coeficientes de la serie de Fourier de una onda cuadrada y período $T = 1$.
    \end{enumerate}
\end{document}