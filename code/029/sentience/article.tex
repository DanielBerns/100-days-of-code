\documentclass[10pt,a4paper]{article}
\usepackage[utf8]{inputenc}
\usepackage[T1]{fontenc}
\usepackage{graphicx}
\usepackage{url}
\author{Observatorio FCE}
\title{The importance of being sentient}
\begin{document}
	\section{The importance of being sentient}
	
	A sentient being perceives and interacts with the world in a subjective manner. This inherent ability grants individuals a private and unique experience, enabling them to develop a complex understanding of their existence. The interplay between an individual's sentience and the environment is a dynamic relationship, giving rise to an array of variable outcomes. Factors such as upbringing, cultural influences, genetic predispositions, and particular experiences further shape the sentient experience, leading to a diverse spectrum of interpretations and responses.
	The consequences of sentience are not always positive. For example, a sentient being may experience fear or anxiety when it is in a dangerous environment. It may also feel pain or suffering when it is injured or ill. In general, we should avoid causing unnecessary suffering to sentient beings whenever possible. 
	
	Human beings are sentient beings. There are other non human sentient beings. 
	
	See Singer's Utilitarian Theory.
	
	Rivers are sentient beings.
	
	% https://www.animallaw.info/article/animal-rights-theory-and-utilitarianism-relative-normative-guidance#:~:text=Singer%20is%20an%20act%20utilitarian,of%20which%20consequences%20are%20relevant.
	
% This article examines Peter Singer’s animal ethic’s theory and argues that the utilitarian calculus’ inherent process of abstraction and homogenisation is epistemically violent because it erases animals’ singularities. I also argue that considering the sentience we can know of as the only characteristic that marks animals as worthy of moral considerability, as Singer does, can lead to violent actions towards animals because this logic erases all the violence that escapes sentientist logics. I show that key to this critique is Singer’s misunderstanding of human sovereignty, and the relationship between human sovereignty and subjectivity. Further, I examine Singer’s conception of the “I”, and find that it is a lifeless and static one that leads his theory to foreclose ethical judgements. This article shows that animals’ irreducibility, vulnerabilities and otherness are sufficient to regard animals as worthy of moral considerability. Finally, I examine some practical implications of the arguments I advance.
	
    The works (see \cite{bekoff2009encyclopedia} and \cite{fine2015forward}) states that a wide range of animals are sentient beings. 

	
	“The issue of animal sentience has implications for all areas of human-animal interaction; if animals can have feelings, as we know many can, both their physical and mental welfare needs must be taken into account. This is very important with respect to laws, policies and people's behavior relating to animals and their welfare.
	
	RSPCA: We're the Royal Society for the Prevention of Cruelty to Animals (RSPCA) and we've been here for animals since 1824. We're the world's oldest and largest animal welfare charity, with the primary focus of rescuing, rehabilitating and rehoming or releasing animals across England and Wales.
	
	Science shows us that many species, not just mammals and birds, should be considered sentient. It wasn't long ago that there was a widely-held view that fish don't feel pain, but ground-breaking research found they can. There is currently debate about whether species like decapod crustaceans (crabs, lobsters etc.) and cephalopods (octopus, squid etc.) are sentient. The RSPCA and many others believe that there is sufficient scientific evidence to indicate that these animals should be considered to be sentient, and therefore protected appropriately by legislation. This would help ensure they are no longer subjected to some of the current practices, like boiling crabs and lobster alive, that cause serious pain and distress.
	
	The RSPCA and others are hoping that there will soon be legislation enshrining the concept of animal sentience in law, so that all government departments would have to pay proper regard to (i.e. consider the impact on) the welfare of sentient animals when developing any policies in any area of life. With a Sentience Bill currently in draft form, some progress towards recognising animal sentience has been made.”
	
	Observation: Sentient beings have legal rights.
	
	The RSPCA has suggested that the definition of sentience should be along the following lines: “Sentience is the capacity to have positive or negative experiences such as pain, distress or pleasure.” 
	RSPCA proposed also that the Bill should include further explanation and guidance, including these concepts:
	\begin{enumerate}
	\item For an animal to be sentient, the nervous system would have to be complex enough to process sensory inputs and create a subjective (or conscious) experience. For example, input from pain sensing nerves would be processed and experienced as suffering and distress.
	\item If endorphins (morphine-like chemicals) are released by an animal in response to pain, it can be inferred that pain is a problem for the animal, and therefore the animal is aware of (experiencing) the pain and is suffering.
	\item Behaviours that indicate pain/suffering (such as a dog yelping), or joy/pleasure (such as rats ‘laughing’ in response to tickling by humans, and actively seeking the experience1), show that animals are having negative or positive experiences, and are therefore sentient.
	\item Sentient animals can be aware of pain, distress and pleasurable feelings without necessarily being able to reflect on these feelings in the same way as humans.
	\item Not all animals meet the criteria for sentience set out above, but the number of species
	regarded as sentient may increase, as new scientific discoveries are made about the physiology and behaviour of invertebrates. 
    \end{enumerate}

\bibliography{article}
\bibliographystyle{plain}
\end{document}